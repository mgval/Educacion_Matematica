\documentclass[10pt]{article}
%\usepackage[utf8]{inputenc}
\oddsidemargin=0cm
\textwidth=15cm
\usepackage{graphicx}
%\usepackage[utf8x]{inputenc}
%\usepackage[spanish]{babel}
%\DeclareGraphicsExtensions{.bmp,.jpg}
\usepackage[latin1]{inputenc}
\usepackage{amsmath}
\usepackage{amsthm}
\usepackage{amsfonts}
\usepackage{adjustbox}
\usepackage{amssymb}
\usepackage[dvips]{epsfig}
\usepackage{ulem}
\usepackage{indentfirst}
\usepackage{wasysym}
\usepackage{pifont}
\usepackage{fancyhdr}
%\usepackage{color}
\usepackage{multicol}
\usepackage{framed}
\usepackage[usenames, dvipsnames]{color}
\usepackage{wrapfig}

%\textwidth=17cm %ancho del texto, de paso define margen de la derecha
%\topmargin=2cm %margen superior
%\oddsidemargin=-0.5cm %margen de la izquierda del texto
%\evensidemargin=0.5cm 

%\definecolor{color1}{RGB}{220,250,250}
\definecolor{shadecolor}{RGB}{220,250,250}
%238
\pagestyle{myheadings}
\definecolor{color1}{RGB}{220,250,250}
\definecolor{color2}{rgb}{0.99,0.9,1.0}
\definecolor{color3}{RGB}{220,250,250}
\definecolor{color4}{rgb}{0.0,0.5,0.69}
\definecolor{color5}{rgb}{1.0,1.0,0.88}
\definecolor{color6}{rgb}{1.0,0.94,0.84}
\definecolor{color7}{rgb}{0.94,1.0,1.0}
\definecolor{color8}{rgb}{1.0,0.94,0.84}
\definecolor{color9}{rgb}{0.0,0.5,0.69}
\definecolor{dblue}{rgb}{1.0,0.0,1.0}
\definecolor{dred}{rgb}{1.0,0.0,0.0}
\definecolor{lred}{rgb}{0.82,0.1,0.26}

%HIGHLIGHT 2 (turquise)
\newcommand{\2}[1]{\hspace{-0.93cm}\colorbox{color1}{\hspace{0.07cm} \parbox{17cm}{\vspace{0.2cm} #1}\hspace*{0.07cm} }}

%HIGHLIGHT 3 ()
\newcommand{\3}[1]{\hspace{-0.93cm}\colorbox{color7}{\hspace{0.07cm} \parbox{17cm}{\vspace{0.2cm} #1}\hspace*{0.07cm} }}
%NOTA
\newcommand{\nota}[1]{\textbf{Notaci\'on:}\: 
#1}

%LINE
\newcommand{\start}{\noindent {\color{color4}\rule{17cm}{0.5mm}}\\}

%Notation
\newcommand{\notation}[1]{\begin{framed}\noindent \nota{#1} \end{framed}}

%COMIENZO
\newcommand{\com}[1]{\noindent\rule{17cm}{0.8mm}
\begin{center}
\textbf{{\Large #1}}
\end{center}
\noindent\rule{17cm}{0.8mm}\\
}

%DEMOSTRACION
\newcommand{\dem}{\noindent {\color{color4}\rule{17cm}{0.5mm}}\\ \negrita{\textbf{Demostraci\'on: }}} 


%REFLEXION
\newcommand{\reflexion}[1]{\2{\textbf{\underline{Reflexi\'on.}}\\

#1\\}\\ }

%NEGRITA
\newcommand{\negrita}[1]{{\color{color4}\textbf{#1}}}

\newcommand{\red}[1]{{\color{red}\textbf{#1}}}

%HIGHLIGHT
\newcommand{\highlight}[1]{\begin{shaded} #1 \end{shaded}}

%FRAME
\newcommand{\enmarcar}[1]{\begin{framed} #1 \end{framed}}


%OBSERVATION 
\newcommand{\obs}[1]{\textbf{Observati\'on:} #1}

%HIGHLIGHT 2
\newcommand{\Highlight}[1]{\hspace{-0.93cm}\colorbox{color9}{\hspace{0.07cm} \parbox{12.7cm}{\vspace{0.2cm} #1}\hspace*{0.07cm} }}

% DESAFIO
\newcommand{\Desafio}[1]{\noindent\rule{13.3cm}{0.4mm}\\
\Highlight{ #1
\vspace{0.3cm} }
\noindent\rule{13.3cm}{0.3mm}\\}

%OBSERVATION
\newcommand{\observation}[1]{\begin{framed} \obs{#1} \end{framed}}

%SOLUTION
\newcommand{\sol}{\noindent {\color{color4}\rule{13.3cm}{0.5mm}}\\ \negrita{\textbf{Soluci\'on: }}} 

%\markboth{}{Construcci\'on de Tesis}

%\title{Control 1 de Matem\'aticas 1}
%\author{Programa de Bachillerato. Universidad de Chile.}
%\date{Lunes 24 de Marzo, 2013}
\theoremstyle{theorem}
\newtheorem{teo}{\textbf{Teorema}}%[chapter]
\newtheorem{lema}{\textbf{Lema}}%
\newtheorem{defi}{\textbf{Definici\'on}}%[chapter]
\newtheorem{propi}{Propiedad}%[chapter]
\newtheorem{propir}{Propiedad Relevante}%[chapter]
\newtheorem{pro}{Proposici\'on}%[chapter]
\newtheorem{ax}{Axioma}
\newtheorem{prob}{Problema}
\newtheorem{eje}{Ejemplo}%[chapter]
\newtheorem{ejer}{\textsc{Ejercicio}}%[section]
\newtheorem{ejerr}{Ejercicio Resuelto}
\newtheorem{obser}{Observaci\'on}%[chapter]
\newcommand{\R}{\mathbb{R}} 
\newcommand{\Z}{\mathbb{Z}}
\newcommand{\Q}{\mathbb{Q}}
\newcommand{\C}{\mathbb{C}}
\newcommand{\N}{\mathbb{N}}
\newcommand{\U}{\mathbb{U}}
\newcommand{\D}{\mathbb{D}}
\newcommand{\I}{\mathbb{I}}
\numberwithin{equation}{section}
\pagestyle{myheadings}
\usepackage{enumerate}
\newcommand{\dis}{\displaystyle}
\usepackage[all]{xy}
\usepackage{tabu}  %paquete para hacer tablas

\title{C\'alculo Diferencial (MAT170)\\ Clase 9 }
\author{Prof. Marco Godoy\\
marco.godoy@edu.udla.cl}
\date{Mayo 2019}

\begin{document}


\maketitle


\begin{enumerate}[1.]
    \item \textbf{El problema de la recta tangente}. Para una funci\'on $y=f(x)$ y $a\in \mathrm{dom}(f)$, decimos que $f$ es derivable (o diferenciable) en $x=a$ si el l\'imite \begin{equation}
    \lim_{h\to 0}\frac{f(a+h)-f(a)}{h}
    \end{equation}
 existe. En ese caso, denotamos por $f'(a)$ dicho l\'imite. La recta tangente de $f$ en el punto $(a,f(a))$ es $y=f'(a)(x-a)+f(a)$. 
    \item[] \textbf{Problema 1}. Para cada una de las siguientes funciones, halle el valor de $\dis f'(x)=\lim_{h\to 0}\frac{f(x+h)-f(x)}{h}$.
  \begin{tabbing}
\hspace{5 cm}\=\hspace{5 cm}\=\kill
 a.  $f(x)=2x-3$ \>  b.  $f(x)=x^2-3x+1$ \> c.  $\dis f(x)=\frac{2x}{x+1}$\\   
\end{tabbing}
    \item[] \textbf{Problema 2}. En cada uno de los casos del Problema 1, calcule la recta tangente en al menos un punto del gr\'afico de cada funci\'on.
    \item \textbf{La funci\'on derivada}. Para una funci\'on $f$, podemos definir la funci\'on derivada $f'$, donde el valor $f'(x)$ es la pendiente de la recta tangente de f en el punto $(x,f(x))$. 
    \item[] \textbf{Problema 3}. En cada uno de los casos del Problema 1, determine la funci\'on derivada. En el caso de que se puede, determine su dominio.
    \item \textbf{Propiedades algebraicas de la derivada}. La derivada es compatible con las operaciones de suma y producto de funciones, en el siguiente sentido:
    \begin{enumerate}[a.]
        \item Si $f$ y $g$ son derivables en $a$, entonces $f+g$ tambi\'en es derivable en $x=a$ y se tiene que $(f+g)'(a)=f'(a)+g'(a)$.
        \item Si $f$ y $g$ son derivables en $a$, entonces el producto $fg$ tambi\'en es derivable en $x=a$ y se tiene que $(fg)'(a)=f'(a)g(a)+f(a)g'(a)$.
        \item Si $f$ y $g$ son derivables en $a$ y $g(a)\neq 0$, entonces $\displaystyle \frac{f}{g}$ es derivable en $a$ y $\displaystyle \frac{f}{g}(a)=\frac{f'(a)g(a)-f(a)g'(a)}{g(a)^2}$.
    \end{enumerate}
    En particular, se pueden estudiar las funciones derivadas $(f+g)'$, $(fg)'$ y $\displaystyle \left(\frac{f}{g}\right)'$, donde:
    \begin{enumerate}[a.]
        \item $(f+g)'=f'+g'$.
        \item $(fg)'=f'g+fg'$. 
        \item $\displaystyle \left(\frac{f}{g}\right)'=\frac{f'g-fg'}{g^2}$.
    \end{enumerate}
       \item \textbf{Derivadas de funciones algebraicas}. Como los polinomios son funciones que consisten en combinaciones de sumas y productos, entonces f\'acilmente se pueden conocer sus derivadas, en particular son derivables en todos los puntos de su dominio. Para ello, s\'olo basta saber que 
    \begin{equation*}
        (x^n)'=nx^{n-1}, 
    \end{equation*} para todo $n\in \mathbb{N}$. Como ejemplo, para la funci\'on polinomial $f(x)=3x^3+x^2-5x+1$, su derivada en $x$ es $f'(x)=9x^2+2x-5x+1$. Un ejemplo menos trivial es calcular la derivada de la funci\'on racional $\displaystyle g(x)=\frac{x^2}{x^3-4x}$,
    \begin{align*}
      g'(x)&=\frac{(x^2)'(x^3-4x)-(x^2)(x^3-4x)'}{(x^3-4x)^2}\\
           &=\frac{2x(x^3-4x)-(x^2)(3x^2-4)}{(x^3-4x)^2}\\
           &=\frac{2x^3-8x^2-(3x^4-4x^2)}{(x^3-4x)^2}\\
           &=\frac{-3x^4+2x^3-12x^2}{(x^3-4x)^2}\\
    \end{align*}           


La igualdad $(x^n)'=nx^{n-1}$ se puede extender a cualquier valor $r\in\mathbb{R}$ no nulo. Es decir:
    \begin{equation*}
        (x^r)'=rx^{r-1}. 
    \end{equation*}
La igualdad anterior es \'util a la hora de calcular derivadas que involucran raices. Como ejemplo, para la funci\'on $h(x)=\sqrt[5]{x^2}$, su derivada es $h'(x)=\displaystyle \frac{2}{5}x^{-\frac{3}{5}}$.
   \item[] \textbf{Problema 4}. Calcule las derivadas del Problema 1 ocupando las propiedades algebraicas de la derivada.
   \item \textbf{Observaci\'on}. Otras notaciones para la derivada en $x=a$ son: $$\frac{df}{dx}(a),\quad D_xf(a).$$
   \item \textbf{Derivadas de funciones trigonom\'etricas}. Tenemos lo siguiente:
   \begin{align*}
   (\sin(x))'&=\cos(x)\\
   (\cos(x))'&=-\sin(x)\\
   (\tan(x))'&=\sec^2(x)\\
   (\sec(x))'&=\sec(x)\tan(x)\\
   (\csc(x))'&=-\cot(x)\csc(x)\\
   (\cot(x))'&=-\csc^2(x)\\
   \end{align*}
   \item \textbf{Derivadas de las funciones exponencial y logaritmo natural}. Tenemos lo siguiente:
   \begin{align*}
   (e^x)'&=e^x\\
   (\ln|x|)'&=\frac{1}{x}
\end{align*}    
\end{enumerate}


\end{document}
