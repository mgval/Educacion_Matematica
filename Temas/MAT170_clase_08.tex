\documentclass[10pt]{article}
%\usepackage[utf8]{inputenc}
\oddsidemargin=0cm
\textwidth=15cm
\usepackage{graphicx}
%\usepackage[utf8x]{inputenc}
%\usepackage[spanish]{babel}
%\DeclareGraphicsExtensions{.bmp,.jpg}
\usepackage[latin1]{inputenc}
\usepackage{amsmath}
\usepackage{amsthm}
\usepackage{amsfonts}
\usepackage{adjustbox}
\usepackage{amssymb}
\usepackage[dvips]{epsfig}
\usepackage{ulem}
\usepackage{indentfirst}
\usepackage{wasysym}
\usepackage{pifont}
\usepackage{fancyhdr}
%\usepackage{color}
\usepackage{multicol}
\usepackage{framed}
\usepackage[usenames, dvipsnames]{color}
\usepackage{wrapfig}

%\textwidth=17cm %ancho del texto, de paso define margen de la derecha
%\topmargin=2cm %margen superior
%\oddsidemargin=-0.5cm %margen de la izquierda del texto
%\evensidemargin=0.5cm 

%\definecolor{color1}{RGB}{220,250,250}
\definecolor{shadecolor}{RGB}{220,250,250}
%238
\pagestyle{myheadings}
\definecolor{color1}{RGB}{220,250,250}
\definecolor{color2}{rgb}{0.99,0.9,1.0}
\definecolor{color3}{RGB}{220,250,250}
\definecolor{color4}{rgb}{0.0,0.5,0.69}
\definecolor{color5}{rgb}{1.0,1.0,0.88}
\definecolor{color6}{rgb}{1.0,0.94,0.84}
\definecolor{color7}{rgb}{0.94,1.0,1.0}
\definecolor{color8}{rgb}{1.0,0.94,0.84}
\definecolor{color9}{rgb}{0.0,0.5,0.69}
\definecolor{dblue}{rgb}{1.0,0.0,1.0}
\definecolor{dred}{rgb}{1.0,0.0,0.0}
\definecolor{lred}{rgb}{0.82,0.1,0.26}

%HIGHLIGHT 2 (turquise)
\newcommand{\2}[1]{\hspace{-0.93cm}\colorbox{color1}{\hspace{0.07cm} \parbox{17cm}{\vspace{0.2cm} #1}\hspace*{0.07cm} }}

%HIGHLIGHT 3 ()
\newcommand{\3}[1]{\hspace{-0.93cm}\colorbox{color7}{\hspace{0.07cm} \parbox{17cm}{\vspace{0.2cm} #1}\hspace*{0.07cm} }}
%NOTA
\newcommand{\nota}[1]{\textbf{Notaci\'on:}\: 
#1}

%LINE
\newcommand{\start}{\noindent {\color{color4}\rule{17cm}{0.5mm}}\\}

%Notation
\newcommand{\notation}[1]{\begin{framed}\noindent \nota{#1} \end{framed}}

%COMIENZO
\newcommand{\com}[1]{\noindent\rule{17cm}{0.8mm}
\begin{center}
\textbf{{\Large #1}}
\end{center}
\noindent\rule{17cm}{0.8mm}\\
}

%DEMOSTRACION
\newcommand{\dem}{\noindent {\color{color4}\rule{17cm}{0.5mm}}\\ \negrita{\textbf{Demostraci\'on: }}} 


%REFLEXION
\newcommand{\reflexion}[1]{\2{\textbf{\underline{Reflexi\'on.}}\\

#1\\}\\ }

%NEGRITA
\newcommand{\negrita}[1]{{\color{color4}\textbf{#1}}}

\newcommand{\red}[1]{{\color{red}\textbf{#1}}}

%HIGHLIGHT
\newcommand{\highlight}[1]{\begin{shaded} #1 \end{shaded}}

%FRAME
\newcommand{\enmarcar}[1]{\begin{framed} #1 \end{framed}}


%OBSERVATION 
\newcommand{\obs}[1]{\textbf{Observati\'on:} #1}

%HIGHLIGHT 2
\newcommand{\Highlight}[1]{\hspace{-0.93cm}\colorbox{color9}{\hspace{0.07cm} \parbox{12.7cm}{\vspace{0.2cm} #1}\hspace*{0.07cm} }}

% DESAFIO
\newcommand{\Desafio}[1]{\noindent\rule{13.3cm}{0.4mm}\\
\Highlight{ #1
\vspace{0.3cm} }
\noindent\rule{13.3cm}{0.3mm}\\}

%OBSERVATION
\newcommand{\observation}[1]{\begin{framed} \obs{#1} \end{framed}}

%SOLUTION
\newcommand{\sol}{\noindent {\color{color4}\rule{13.3cm}{0.5mm}}\\ \negrita{\textbf{Soluci\'on: }}} 

%\markboth{}{Construcci\'on de Tesis}

%\title{Control 1 de Matem\'aticas 1}
%\author{Programa de Bachillerato. Universidad de Chile.}
%\date{Lunes 24 de Marzo, 2013}
\theoremstyle{theorem}
\newtheorem{teo}{\textbf{Teorema}}%[chapter]
\newtheorem{lema}{\textbf{Lema}}%
\newtheorem{defi}{\textbf{Definici\'on}}%[chapter]
\newtheorem{propi}{Propiedad}%[chapter]
\newtheorem{propir}{Propiedad Relevante}%[chapter]
\newtheorem{pro}{Proposici\'on}%[chapter]
\newtheorem{ax}{Axioma}
\newtheorem{prob}{Problema}
\newtheorem{eje}{Ejemplo}%[chapter]
\newtheorem{ejer}{\textsc{Ejercicio}}%[section]
\newtheorem{ejerr}{Ejercicio Resuelto}
\newtheorem{obser}{Observaci\'on}%[chapter]
\newcommand{\R}{\mathbb{R}} 
\newcommand{\Z}{\mathbb{Z}}
\newcommand{\Q}{\mathbb{Q}}
\newcommand{\C}{\mathbb{C}}
\newcommand{\N}{\mathbb{N}}
\newcommand{\U}{\mathbb{U}}
\newcommand{\D}{\mathbb{D}}
\newcommand{\I}{\mathbb{I}}
\numberwithin{equation}{section}
\pagestyle{myheadings}
\usepackage{enumerate}
\newcommand{\dis}{\displaystyle}
\usepackage[all]{xy}
\usepackage{tabu}  %paquete para hacer tablas

\title{C\'alculo Diferencial (MAT170)\\ Clase 8 }
\author{Prof. Marco Godoy\\
marco.godoy@edu.udla.cl}
\date{Mayo 2019}

\begin{document}


\maketitle

\section{L\'imites y continuidad}

\subsection{Ejemplos b\'asicos para entender la idea de l\'imite}

\begin{enumerate}[P1.]
\item Realice una tabla de valores $x$ versus $ f(x)$ (que sean cercanos) con las funciones que se indican a continuaci\'on e indique qu\'e puede concluir al respecto.
  \begin{enumerate}[1.]
     \item $\dis f(x)=\frac{x^4-16}{x^2-4}$ en torno al punto $x=2$.\\
     \item[] \textbf{Desarrollo} Lo importante es obtener los valores de $f(x)$ cuando $x$ es un valor muy cercano (pero no necesariamente igual) a $2$:
     \begin{table}[h!]
   \centering
   \begin{tabular}{|c |c| c| c| c| c| c| c| c| c |} 
   \hline
     $x$   &  $1.5$ & $1.6$&$1.7$&$1.8$&$1.9$&$2.1$&$2.2$ &$2.3$&$2.4$  \\ %[1ex] 
   \hline
   $f(x)$ &   $6.25$ & $6.56$&$6.89$&$7.24$&$7.61$&$8.41$&$8.84$ &$9.29$&$9.76$  \\%[1ex] 
   \hline
   \end{tabular}
 %\caption{Table to test captions and labels}
 %\label{table:1}
   \end{table}
   \\
   Podemos hacer esta tabla m\'as representativa para estudiar el problema:
   \begin{table}[h!]
   \centering
   \begin{tabular}{|c |c| c| c| c| c| c| c| c| c |} 
   \hline
     $x$   &  $1.9$ & $1.93$&$1.96$&$1.99$&$2.01$&$2.04$&$2.07$ &$2.10$&$2.13$  \\ %[1ex] 
   \hline
   $f(x)$ &   $7.61$ & $7.7249$&$7.8416$&$7.9601$& $8.0401$&$8.1616$&$8.2849$ &$8.41$&$8.5369$  \\%[1ex] 
   \hline
   \end{tabular}
 %\caption{Table to test captions and labels}
 %\label{table:1}
   \end{table}
   En resumen, mientras m\'as cerca est\'e $x$ del $2$, m\'as cerca estar\'a el valor $f(x)$ del 8.% En otras palabras: $$ \lim_{x\to 2}f(x)=8$$
     \item $\dis g(x)=\frac{\sqrt{x+3}-\sqrt{3}}{x}$ en torno al punto $x=0$.
     \item[] \textbf{Desarrollo.}
     Procedemos con el mismo estilo de tabla que en el punto anterior:
   \begin{table}[h!]
   \centering
   \begin{tabular}{|c |c| c| c| c| c| c| c| c| } 
   \hline
     $x$   &  $-0.1$ & $-0.07$&$-0.04$&$-0.01$&$0.01$&$0.04$&$0.07$ &$0.10$  \\ %[1ex] 
   \hline
   $g(x)$ &   $0.2911$ & $0.2903$&$0.2896$&$0.2889$& $0.2884$&$0.2877$&$0.287$ &$0.2863$  \\%[1ex] 
   \hline
   \end{tabular}
 %\caption{Table to test captions and labels}
 %\label{table:1}
   \end{table}
    En resumen, mientras m\'as cerca est\'e $x$ del $0$, m\'as cerca estar\'a el valor $f(x)$ de un n\'umero que se encuentra dentro del intervalo $[0.2884,0.2889]$. Si se quiere obtener m\'as informaci\'on, hay que hacer una tabla con valores m\'as pr\'oximos a $0$.
     \item $\dis h(x)=\frac{|x-5|}{x-5}$ en torno al punto $x=5$.
  \end{enumerate}
\end{enumerate}

Los siguientes problemas propuestos fueron extra\'idos de la C\'atedra II de curso MAT170 a\~no 2013.

\begin{enumerate}[1.]
  \item Calcule $\dis \lim_{x\to 2}\frac{x^4-16}{x^2-4}$.
  \item[] \textbf{Desarrollo.} Lo que importa es que siempre se asume que $x\neq 2$, por lo que los siguientes c\'alculos est\'an perfectamente justificados:
  \begin{align*}
  \lim_{x\to 2}\frac{x^4-16}{x^2-4}&=\lim_{x\to 2}\frac{(x^2-4)(x^2+4)}{x^2-4}\\
                                   &=\lim_{x\to 2}x^2+4\\
                                   &=2^2+4\\
                                   &=8
\end{align*}   Compare este problema con el problema de la tabla de valores de la misma funci\'on.
  \item Calcule $\dis \lim_{x\to 2}\frac{x-1}{\sqrt[3]{6x+2}-2}$ (Observaci\'on: Usar la variable auxiliar $u=\sqrt[3]{6x+2}$).
  \item[] \textbf{Desarrollo.} Si $u=\sqrt[3]{6x+2}$ entonces podemos despejar $x$ de la igualdad: $$x=\frac{u^3-2}{6}$$
  Reemplazando:
  \begin{align*}
  \frac{x-1}{\sqrt[3]{6x+2}-2}&=\frac{\frac{u^3-2}{6}-1}{u-2}\\
                              &=\frac{u^3-8}{6(u-2)}\\
                              &=\frac{(u-2)(u^2+2u+4)}{6(u-2)}\\
                              &=\frac{u^2+2u+4}{6}
  \end{align*} O sea, la funci\'on $\dis \frac{x-1}{\sqrt[3]{6x+2}-2}$ fue convertida en la funci\'on $\dis \frac{u^2+2u+4}{6}$ gracias al cambio de variable. Por otro lado, $$ \lim_{x\to 2}u(x)= \lim_{x\to 2}\sqrt[3]{6x+2}=\sqrt[3]{\lim_{x\to 2} 6x+2}=\sqrt[3]{14}.$$ Es decir, si $x$ converge a $2$, $u$ converge a $\sqrt[3]{14}$. En lenguaje matem\'atico: 
  \begin{center}
  $u\to \sqrt[3]{14}$ cuando $x\to 2$.
  \end{center}
  Aplicando lo anterior al problema:
  \begin{align*}
  \lim_{x\to 2}\frac{x-1}{\sqrt[3]{6x+2}-2}&=\lim_{u\to \sqrt[3]{14}}\frac{u^2+2u+4}{6}\\
                                           &=\frac{1}{6}\lim_{u\to \sqrt[3]{14}}(u^2+2u+4)\\
                                           &=\frac{1}{6}\left( \sqrt[3]{14^2}+2\sqrt[3]{14}+4 \right)\\
                                           &=\frac{\sqrt[3]{14^2}}{6}+\frac{\sqrt[3]{14}}{3}+\frac{2}{3} 
  \end{align*}
  \item Analice la continuidad de $G$ en $x=2$ si $$G(x)=\left\lbrace  \begin{array}{ccc}
\dis \frac{x^2+2x}{x},&si& x\leq 2\\ \dis \frac{x-2}{\sqrt{x+2}-2}  ,&si& x>2 
\end{array}        \right.$$
  \item[]\textbf{Desarrollo.} Se debe demostrar que $\dis \lim_{x\to 2}G(x)=G(2)$.
  \begin{itemize}
    \item Por definici\'on de la funci\'on $G$, es f\'acil verificar que $G(2)=4$.
    \item La funci\'on $G$ tiene dos comportamientos distintos, dependiendo de si $x$ se encuentra a la izquierda del $2$ o a la derecha. La mejor manera de estudiar $\dis \lim_{x\to 2}G(x)$ es mediante l\'imites laterales:
    \begin{align*}
    \dis \lim_{x\to 2^+}G(x)&=\lim_{x\to 2^+}\frac{x-2}{\sqrt{x+2}-2}\\
                            &=\lim_{x\to 2^+}\frac{(x-2)(\sqrt{x+2}+2)}{(\sqrt{x+2}-2)(\sqrt{x+2}+2)}\\
                            &=\lim_{x\to 2^+}\frac{(x-2)(\sqrt{x+2}+2)}{(x+2-4)}\\
                            &=\lim_{x\to 2^+}\frac{(x-2)(\sqrt{x+2}+2)}{(x-2)}\\
                            &=\lim_{x\to 2^+}\sqrt{x+2}+2\\
                            &=\sqrt{\lim_{x\to 2^+}(x+2)}+2\\
                            &=\sqrt{4}+2\\
                            &=4.
    \end{align*}
    \begin{align*}
    \dis \lim_{x\to 2^-}G(x)&=\lim_{x\to 2^-}\frac{x^2+2x}{x}\\
                            &=\frac{\lim_{x\to 2^-}(x^2+2x)}{\lim_{x\to 2^-}x}\\
                            &=\frac{4+4}{2}\\
                            &=4
    \end{align*}        
    Como los l\'imites laterales son iguales, se cumple que $\dis\lim_{x\to 2}G(x)$ existe y $\dis\lim_{x\to 2}G(x)=4$. Al cumplirse $\dis\lim_{x\to 2}G(x)=G(2)$, se finaliza con que la funci\'on $G$ es continua en $2$.
  \end{itemize}
\end{enumerate}

Los siguientes problemas propuestos fueron extra\'idos de la C\'atedra II de curso MAT170 NRC:11166 a\~no 2014.
\begin{enumerate}[1.]
  \item Calcule $\dis \lim_{x\to 1}\frac{7x^2-4x-3}{x-1}$.
  \item[] \textbf{Desarrollo.}
  Si se eval\'ua directamente el l\'imite en el numerador y denominador, tal como se hizo en clases, nos da un l\'imite del tipo $\dis \frac{0}{0}$. Luego hay que hacer un trabajo algebraico adicional.\\
  Intentamos factorizar el numerador. Notar que $7x^2-4x-3=0$ cuando $\dis x=1,-\frac{3}{7}$. Luego: $$7x^2-4x-3=7(x-1)\left(x+\frac{3}{7}\right).$$
As\'i:  
  \begin{align*}
  \lim_{x\to 1}\frac{7x^2-4x-3}{x-1}&=\lim_{x\to 1}\frac{7(x-1)\left(x+\frac{3}{7}\right)}{x-1}\\
                                    &=\lim_{x\to 1}7\left(x+\frac{3}{7}\right)\\
                                    &=\lim_{x\to 1}7x+3\\
                                    &=10
  \end{align*}
  \item Calcule $\dis \lim_{x\to 0}\frac{\sqrt{x+2}-\sqrt{2}}{x}$.
  \item[] Cuando aparece una diferencia de ra\'ices cuadradas en un l\'imite, se sugiere agregar el factor para obtener una suma por diferencia: 
  \begin{align*}
   \lim_{x\to 0}\frac{\sqrt{x+2}-\sqrt{2}}{x}&=\lim_{x\to 0}\frac{(\sqrt{x+2}-\sqrt{2})(\sqrt{x+2}+\sqrt{2})}{x(\sqrt{x+2}+\sqrt{2})}\\
   &=\lim_{x\to 0}\frac{x+2-2}{x(\sqrt{x+2}+\sqrt{2})}\\
   &=\lim_{x\to 0}\frac{x}{x(\sqrt{x+2}+\sqrt{2})}\\
   &=\lim_{x\to 0}\frac{1}{\sqrt{x+2}+\sqrt{2}}\\
   &=\frac{\lim_{x\to 0} 1}{\lim_{x\to 0}(\sqrt{x+2}+\sqrt{2})}\\
   &=\frac{ 1}{(\sqrt{\lim_{x\to 0}(x+2)}+\sqrt{2})}\\
   &=\frac{1}{\sqrt{2}+\sqrt{2}}\\
   &=\frac{1}{2\sqrt{2}}\\
   &=\frac{\sqrt{2}}{4}.
  \end{align*}
  \item Encontrar los valores de $a$ y $b$ para que la funci\'on sea continua en todo el conjunto $\R$: $$f(x)=\left\lbrace  \begin{array}{ccc}
4x-2a,&si& x\leq -1\\ 3ax+2b  ,&si& -1\leq x\leq 2 \\2x-6b,&si& x> 2
\end{array}        \right.$$
  \item[] \textbf{Desarrollo.} Vemos lo siguiente:
  \begin{enumerate}[1.]
     \item Si $x\in ]-\infty,-1[$, entonces $f(x)=4x-2a$. Como es un polinomio, entonces $f$ es continua en todos los puntos de $]-\infty,-1[$.
     \item Si $x\in ]-1,2[$, entonces $f(x)=3ax+2b$. Como es un polinomio, entonces $f$ es continua en todos los puntos de $]-1,2[$.
     \item Si $x\in ]2,\infty[$, entonces $f(x)=2x-6b$. Como es un polinomio, entonces $f$ es continua en todos los puntos de $]2,\infty[$.
  \end{enumerate}
  O sea, en los \'unicos puntos donde es necesario demostrar que se cumple la condici\'on de continuidad es en $x=-1$ y en $x=2$. Para resolver el problema, hay que seguir de la siguiente manera:
  \begin{enumerate}[1.]
     \item (Continuidad en $x=-1$) Evaluando, $f(-1)=-4-2a$. Los l\'imites laterales son los siguientes: $$\lim_{x\to -1^{-}}f(x)=-4-2a,\qquad  \lim_{x\to -1^{+}}f(x)=-3a+2b$$ Como debe existir $\dis \lim_{x\to -1}f(x)$, la condici\'on es que $\dis \lim_{x\to -1^{-}}f(x)=\lim_{x\to -1^{+}}f(x)=f(-1)$, o de manera equivalente, $-4-2a=-3a+2b$. La primera condici\'on es: $$a-2b=4$$ 
     \item (Continuidad en $x=2$) Evaluando, $f(2)=6a+2b$. Los l\'imites laterales son los siguientes: $$\lim_{x\to 2^{-}}f(x)=6a+2b,\qquad  \lim_{x\to 2^{+}}f(x)=4-6b$$ Como debe existir $\dis \lim_{x\to 2}f(x)$, la condici\'on es que $\dis \lim_{x\to 2^{-}}f(x)=\lim_{x\to 2^{+}}f(x)=f(2)$, o de manera equivalente, $6a+2b=4-6b$. La primera condici\'on es: $$6a+8b=4$$ 
  \end{enumerate}
  Las soluciones del sistema de ecuaciones: 
  \begin{align*}
  a-2b&=4\\
  6a+8b&=4
  \end{align*} son $\dis a=2,\; b=-1$.
\end{enumerate}

\section{Problemas adicionales}

\begin{enumerate}[P1.]
  \item Para cada una de las siguientes funciones, halle el valor de $\dis \lim_{h\to 0}\frac{f(x+h)-f(x)}{h}$.
  \begin{tabbing}
\hspace{5 cm}\=\hspace{5 cm}\=\kill
 a.  $f(x)=2x-3$ \>  b.  $f(x)=x^2-3x+1$ \> c.  $\dis f(x)=\frac{2x}{x+1}$\\   
\end{tabbing}
  \item Determine el valor de $k$, de modo que $\dis \lim_{x\to 4}f(x)$ exista cuando $$f(x)=\left\lbrace  \begin{array}{ccc}
3x+2,&si& x< 4\\ 5x+k  ,&si& x\geq 4 
\end{array}        \right.$$ 
\end{enumerate}



\end{document}